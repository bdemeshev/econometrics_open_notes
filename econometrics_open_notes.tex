% arara: xelatex
\documentclass[12pt]{article}

% \usepackage{physics}

\usepackage{hyperref}
\hypersetup{
    colorlinks=true,
    linkcolor=blue,
    filecolor=magenta,      
    urlcolor=cyan,
    pdftitle={Overleaf Example},
    pdfpagemode=FullScreen,
    }

\usepackage{verse}

\usepackage{tikzducks}

\usepackage{tikz} % картинки в tikz
\usetikzlibrary{arrows.meta}
\usetikzlibrary{shapes, arrows, positioning}
\usepackage{microtype} % свешивание пунктуации

\usepackage{array} % для столбцов фиксированной ширины

\usepackage{indentfirst} % отступ в первом параграфе

\usepackage{sectsty} % для центрирования названий частей
\allsectionsfont{\centering}

\usepackage{amsmath, amsfonts, amssymb} % куча стандартных математических плюшек

\usepackage{comment}

\usepackage[top=2cm, left=1.2cm, right=1.2cm, bottom=2cm]{geometry} % размер текста на странице

\usepackage{lastpage} % чтобы узнать номер последней страницы

\usepackage{enumitem} % дополнительные плюшки для списков
%  например \begin{enumerate}[resume] позволяет продолжить нумерацию в новом списке
\usepackage{caption}

\usepackage{url} % to use \url{link to web}


\newcommand{\smallduck}{\begin{tikzpicture}[scale=0.3]
    \duck[
        cape=black,
        hat=black,
        mask=black
    ]
    \end{tikzpicture}}

\usepackage{fancyhdr} % весёлые колонтитулы
\pagestyle{fancy}
\lhead{}
\chead{}
\rhead{}
\lfoot{}
\cfoot{}
\rfoot{}

\renewcommand{\headrulewidth}{0.4pt}
\renewcommand{\footrulewidth}{0.4pt}

\usepackage{tcolorbox} % рамочки!

\usepackage{todonotes} % для вставки в документ заметок о том, что осталось сделать
% \todo{Здесь надо коэффициенты исправить}
% \missingfigure{Здесь будет Последний день Помпеи}
% \listoftodos - печатает все поставленные \todo'шки


% более красивые таблицы
\usepackage{booktabs}
% заповеди из докупентации:
% 1. Не используйте вертикальные линни
% 2. Не используйте двойные линии
% 3. Единицы измерения - в шапку таблицы
% 4. Не сокращайте .1 вместо 0.1
% 5. Повторяющееся значение повторяйте, а не говорите "то же"


\setcounter{MaxMatrixCols}{20}
% by crazy default pmatrix supports only 10 cols :)


\usepackage{fontspec}
\usepackage{libertine}
\usepackage{polyglossia}

\setmainlanguage{russian}
\setotherlanguages{english}

% download "Linux Libertine" fonts:
% http://www.linuxlibertine.org/index.php?id=91&L=1
% \setmainfont{Linux Libertine O} % or Helvetica, Arial, Cambria
% why do we need \newfontfamily:
% http://tex.stackexchange.com/questions/91507/
% \newfontfamily{\cyrillicfonttt}{Linux Libertine O}

\AddEnumerateCounter{\asbuk}{\russian@alph}{щ} % для списков с русскими буквами
\setlist[enumerate, 1]{label=\asbuk*),ref=\asbuk*}

%% эконометрические сокращения
\DeclareMathOperator{\Cov}{\mathbb{C}ov}
\DeclareMathOperator{\Corr}{\mathbb{C}orr}
\DeclareMathOperator{\Var}{\mathbb{V}ar}
\DeclareMathOperator{\col}{col}
\DeclareMathOperator{\row}{row}
\DeclareMathOperator{\BestLin}{BestLin}

\let\P\relax
\DeclareMathOperator{\P}{\mathbb{P}}

\DeclareMathOperator{\E}{\mathbb{E}}
\DeclareMathOperator{\trace}{trace}
\DeclareMathOperator{\card}{card}
\DeclareMathOperator{\mgf}{mgf}

\DeclareMathOperator{\Convex}{Convex}
\DeclareMathOperator{\plim}{plim}

\usepackage{mathtools}
\DeclarePairedDelimiter{\norm}{\lVert}{\rVert}
\DeclarePairedDelimiter{\abs}{\lvert}{\rvert}
\DeclarePairedDelimiter{\scalp}{\langle}{\rangle}
\DeclarePairedDelimiter{\ceil}{\lceil}{\rceil}

\newcommand{\cN}{\mathcal{N}}
\newcommand{\cF}{\mathcal{F}}

\newcommand{\RR}{\mathbb{R}}
\newcommand{\NN}{\mathbb{N}}

\newcommand{\dBern}{\mathrm{Bern}}
\newcommand{\dPois}{\mathrm{Pois}}
\newcommand{\dBin}{\mathrm{Bin}}
\newcommand{\dMult}{\mathrm{Mult}}
\newcommand{\dGeom}{\mathrm{Geom}}
\newcommand{\dNHGeom}{\mathrm{NHGeom}}
\newcommand{\dHGeom}{\mathrm{HGeom}}
\newcommand{\dDUnif}{\mathrm{DUnif}}
\newcommand{\dFS}{\mathrm{FS}}
\newcommand{\dNBin}{\mathrm{NBin}}

\newcommand{\dTri}{\mathrm{Triangle}}
\newcommand{\dUnif}{\mathrm{Unif}}
\newcommand{\dCauchy}{\mathrm{Cauchy}}
\newcommand{\dN}{\mathcal{N}}
\newcommand{\dLN}{\mathcal{LN}}
\newcommand{\dExpo}{\mathrm{Expo}}
\newcommand{\dBeta}{\mathrm{Beta}}
\newcommand{\dGamma}{\mathrm{Gamma}}
\newcommand{\dWei}{\mathrm{Wei}}
\newcommand{\dLogistic}{\mathrm{Logistic}}
\newcommand{\dRayleigh}{\mathrm{Rayleigh}}
\newcommand{\dPareto}{\mathrm{Pareto}}


\renewcommand{\b}{\beta}
\newcommand{\hb}{\hat{\beta}}
\renewcommand{\u}{u}
\newcommand{\hu}{\hat{u}}
\newcommand{\hy}{\hat{y}}


\newcommand{\alt}{\text{alt}}
\newcommand{\adj}{\text{adj}}
\newcommand{\best}{\text{best}}
\newcommand{\ols}{\text{ols}}

\usepackage{thmtools} % тонкая настройка окружения теорем


% окружения!
\declaretheorem[
style=definition, 
title=Теорема, 
numberwithin=section, 
]{theorem}


% опция sibling=theorem посередине даёт сквозную нумерацию с теоремами
\declaretheorem[
style=definition, 
title=Определение, 
sibling=theorem, 
]{definition}

\declaretheorem[
style=definition, 
title=Задача, 
sibling=theorem, 
]{problem}



%\newenvironment{problem}{}{} % environment just prints the text
\newenvironment{sol}{}{}
\newcommand{\trru}[1]{#1}
\newcommand{\tren}[1]{#1}



\begin{document}

% задачи можно дёргать:
% https://github.com/bdemeshev/metrics_pro/blob/master/metrics_body.tex
% и ещё (слегка модифицируя)
% https://github.com/bdemeshev/em_pset
% картиночки и доказательства на английском:
% https://github.com/olyagnilova/gauss-markov-pythagoras




\section{Методы получения оценок}

Методы получения оценок: метод максимального правдоподобия, метод моментов, метод наименьших квадратов.


\section{Свойства оценок}
Свойства оценок: несмещённость, состоятельность, эффективность в классе.


\section{Асимптотические методы}
Центральная предельная теорема. Лемма Слуцкого. Дельта-метод. Построение асимптотических доверительных интервалов.


\section{Святая троица тестов}
Три классических теста: LM, LR, Wald.


Чёрный трек: тесты в матричной форме для вектора параметров?

\section{ШБ — МНК}
МНК в скалярной и матричной форме без статистических свойств. Строгая мультиколлинеарность. 


Рассказать, что коэффициенты при стандартизации всех переменных называют частными корреляциями. 

Коммент: Здесь первый раз говорим слова "строгая мультиколлинеарность".

Чёрный трэк: нелинейный мнк численно?

Задачи для доски:

МНК и R2 руками на доске

Задачи для колаба:

МНК и R2

Рост R2 c ростом числа регрессоров

Рост RSS с ростом числа наблюдений

\section{Предпосылки о математическом ожидании и дисперсии}
МНК со статистическими предпосылками на ожидание и дисперсию. Теорема Гаусса-Маркова.




\subsection{Ожидание и ковариационная матрица }

Пусть $r$ --- случайный вектор размерности $n \times 1$, $s$ --- случайный вектор размерности $k \times 1$. $A$ и $b$ --- неслучайные матрица и вектор соответственно, имеющие подходящие размерности.

Математическим ожиданием случайного вектора $r$ называется вектор
\[
\E(r) = \begin{pmatrix}
	\E(r_1)  \\
	\E(r_2)  \\
        \dots \\
        \E(r_n)
      \end{pmatrix}.
\]

Ковариационная матрица вектора $r$ определяется следующим образом:
\[
\Var(r) = \begin{pmatrix}
	\Cov(r_1,r_1) & \Cov(r_1,r_2) & \dots & \Cov(r_1,r_n) \\
	\Cov(r_2,r_1) & \Cov(r_2,r_2) & \dots & \Cov(r_2,r_n) \\
        \dots & & \dots & \dots\\
        \Cov(r_n,r_1) & \Cov(r_n,r_2) & \dots & \Cov(r_n,r_n) \\
      \end{pmatrix}.
\]

Ковариационная матрица векторов $r$ и $s$ определяется следующим образом:
\[
\Cov(r,s) = \begin{pmatrix}
	\Cov(r_1,s_1) & \Cov(r_1,s_2) & \dots & \Cov(r_1,s_k) \\
	\Cov(r_2,s_1) & \Cov(r_2,s_2) & \dots & \Cov(r_2,s_k) \\
        \dots & \dots & \dots & \dots\\
        \Cov(r_n,s_1) & \Cov(r_n,s_2) & \dots & \Cov(r_n,s_k) \\
      \end{pmatrix}.
\]

Далее рассмотрим свойства для вектора математических ожиданий и ковариационной матрицы:
\begin{enumerate}
    \item $\E(Ar+b) = A\E(r)+b$
    \item $\Cov(r,s) = \E(rs^T)-\E(r)\E(s^T)$
    \item $\Cov(Ar+b,s) = A\Cov(r,s)$
    \item $\Cov(r,As+b) = \Cov(r,s)A^T$
    \item $\Var(r)=\Cov(r,r) = \E(rr^T)-\E(r)\E(r^T)$
    \item $\Var(Ar+b) = A\Var(r)A^T$
    \item $\E(r^T Ar) = \trace(A\Var(r))+\E(r^T) A\E(r)$  
\end{enumerate}

\subsection{Иерархия зависимостей случайных величин}

Можно выделить три степени независимости случайных величин. 
Рассмотрим их на примере пары произвольных величин $r$ и $s$.

\begin{tikzpicture}[node distance=2cm] % Set node distance
    % Define nodes
    \node[rectangle, draw=orange, line width=1.5pt] (indep) {$r$ и $s$ независимы}; 
    \node[rectangle, draw=orange, line width=1.5pt, right=2cm of indep] (indep2) {$\Cov(h_1(r), h_2(s)) = 0$ для любых функций $h_1$ и $h_2$}; 

    \node[rectangle, draw=orange, line width=1.5pt, below of=indep] (noreg) {$\E(r \mid s) = \E(r)$}; 
    \node[rectangle, draw=orange, line width=1.5pt, right=3cm of noreg] (noreg2) {$\Cov(r, h(s)) = 0$ для любой функции $h$}; 

    \node[rectangle, draw=orange, line width=1.5pt, align=center, below of=noreg] (nocov) {$r$ и $s$ линейно-независимы, \\
    $\BestLin(r \mid s) = \E(r)$}; 
    \node[rectangle, draw=orange, line width=1.5pt, right=3cm of nocov] (nocov2) {$\Cov(r, s) = 0$}; 

    % Connect nodes with arrows
    \draw[{Latex[length=5mm, width=4mm]}-{Latex[length=5mm, width=4mm]}, line width=1pt, double distance=3pt] (indep) -- (indep2);
    \draw[{Latex[length=5mm, width=4mm]}-{Latex[length=5mm, width=4mm]}, line width=1pt, double distance=3pt] (noreg) -- (noreg2);
    \draw[{Latex[length=5mm, width=4mm]}-{Latex[length=5mm, width=4mm]}, line width=1pt, double distance=3pt] (nocov) -- (nocov2);

    \draw[-{Latex[length=5mm, width=4mm]}, line width=1pt, double distance=3pt] (indep) -- (noreg);
    \draw[-{Latex[length=5mm, width=4mm]}, line width=1pt, double distance=3pt] (noreg) -- (nocov);
\end{tikzpicture}

Напомним, что случайные величины $r$ и $s$ называются независимыми, если 
для любых числовых множеств $A$ и $B$ независимы события $\{r \in A\}$ и $\{s \in B\}$:
\[
\P(r \in A, s \in B) = \P(r \in A) \cdot \P(s \in B)
\]

Определить линейную независимость величин $r$ и $s$ можно по-разному. 
Некоторые авторы считают условие $\Cov(r, s) = 0$ определением линейной независимости, в этом случае нижняя эквивалентность на графике тривиальна. 
Нам кажется более логичным другой подход.
Величины $r$ и $s$ линейно независимы, если 
наилучшее линейное приближение $r$ с помощью $s$ не зависит от $s$.

\begin{problem}
    Покажем, что из равенства условного и безусловного математических ожиданий не следует независимость случайных величин:  
    
    \begin{tabular}{|l|l|l|l|}
    \hline
      & 1/3 & 1/3 & 1/3 \\ \hline
    r & -1  & 1   & 0   \\ \hline
    s & 0   & 0   & 1   \\ \hline
    \end{tabular}  
\end{problem}

    

\begin{definition}
Наилучшее линейное приближение величины $r$ с помощью величины $s$ — это линейная функция от $s$,
\[
\BestLin(r \mid s) = \alpha + \beta s,
\]
где константы $\alpha$ и $\beta$ находятся из решения задачи оптимизации
$\E((r - \BestLin(r, s)^2) \to \min_{\alpha, \beta}$.
    
\end{definition}


\subsection{Теорема Гаусса~--- Маркова}

Чтобы исследовать свойства полученной точечной оценки $\hb$ нам потребуются предпосылки о математическом ожидании и ковариационной матрице вектора $\u$.

Мы предположим, что случайные ошибки в среднем равны нулю, а именно,
\[
\E(\u \mid X) = 0.
\]

Предпосылку о математическом ожидании можно записать и в скалярном виде,
\[
\E(\u_i \mid X) = 0, \quad \text{ при } \forall i \in \{1, \dots, n\}.
\]


\begin{theorem}
Если 
\begin{enumerate}
    \item Модель линейна по параметрам: $y = X\b + u$;
    \item Матрица $X$ размера $[n \times k]$ имеет полный ранг $k$.
    \item Условное ожидание ошибок равно нулю, $\E(\u \mid X) = 0$;
    \item Условная ковариационная матрица ошибок пропорциональна единичной, $\Var(\u \mid X) = \sigma^2 I$;
    \item Оценка $\hb$ получена методом наименьших квадратов, $\hb = (X^T X)^{-1}X^T y$;
    \item Альтернативная оценка $\hb^{\alt}$ является условно несмещёнными $\E(\hb^{\alt} \mid X) = \b$
    и линейной по $y$;
\end{enumerate}
то
\begin{enumerate}
    \item Оценка $\hb$ является линейной по $y$;
    \item Оценка $\hb$ является условно несмещённой, $\E(\hb \mid X) = \b$ и несмещённой, $\E(\hb) = \b$;
    \item Оценка любого коэффициента $\hb_j$ является более эффективной, чем альтернативная оценка $\hb_j^{\alt}$:
\[
\Var(\hb_j \mid X) \leq \Var(\hb^{\alt}_j \mid X).
\]
\end{enumerate}
\end{theorem}

Вывод теоремы можно усилить, для любой линейной комбинации коэффициентов $w^T \b$ МНК-оценка $w^T \hb$ эффективнее альтернативной оценки $w^T \hb^{\alt}$:
\[
\Var(w^T\hb_j \mid X) \leq \Var(w^T \hb^{\alt}_j \mid X).
\]


\subsection{Задачи для доски:}

\begin{problem}
Исследовательница Мишель собрала данные по 20 студентам. 
Переменная $y_i$ --- количество решённых задач по эконометрике $i$-ым студентом, 
а $x_i$ --- количество просмотренных серий любимого сериала за прошедший год. 
Оказалось, что $\sum y_i = 10$, $\sum x_i = 0$, $\sum x_i^2 = 40$, $\sum y_i^2 = 50$, $\sum x_i y_i = 60$.

\begin{enumerate}
\item Найдите МНК-оценки коэффициентов парной регрессии.
\item В рамках предположения $\E(u_i \mid X) = 0$ найдите $\E(y_i \mid X)$, $\E(\hb_j \mid X)$, $\E(\hat u_i \mid X)$, $\E(\hat y_i \mid X)$.
\item Предположим дополнительно, что $\Var(u_i \mid X)=\sigma^2$ и $u_i$ при фиксированных $X$ независимы. 
Найдите $\Var(y_i \mid X)$, $\Var(y_i (x_i - \bar x) \mid X)$, $\Var(\sum y_i (x_i - \bar x) \mid X)$, $\Var(\hb_2 \mid X)$.
\end{enumerate}
\begin{sol}

  \end{sol}
\end{problem}

\begin{problem}
Рассмотрим классическую линейную модель $y=X\beta + u$ с предпосылками Гаусса~--- Маркова: $\E(u \mid X) = 0$ и $\Var(u \mid X) = \sigma^2 I$.
Для всех случайных векторов ($y$, $\hy$, $\hb$, $u$, $\hat u$, $\bar y$) найдите все возможные ожидания и ковариационные матрицы
$\E(\cdot)$, $\Var(\cdot)$, $\Cov(\cdot, \cdot)$.
\begin{sol}
\end{sol}
\end{problem}

\begin{problem}
\trru{Рассмотрим модель $y_i = \beta x_i + u_i$ с двумя наблюдениями, $x_1 = 1$, $x_2 = 2$.
Величины $u_1$ и $u_2$ независимы и равновероятно равны $+1$ или $-1$.}


\begin{enumerate}
  \item Найдите оценку $\hat\beta_{\ols}$ для $\beta$ с помощью метода наименьших квадратов. 
  \item Чему равна дисперсия $\Var(\hat\beta_{\text{ols}} \mid x)$ и ожидание $\E(\hat\beta_{\text{ols}} \mid x)$?
  \item Постройте несмещённую оценку $\hat\beta_{\text{best}}$ с наименьшей дисперсией.
  \item Чему равна дисперсия $\Var(\hat\beta_{\text{best}}\mid x)$?
  \item А как же теорема Гаусса — Маркова? Почему в данном примере удаётся построить оценку с дисперсией меньше, чем у оценки методом наименьших квадратов?

\end{enumerate}
  \begin{sol}
    \begin{enumerate}
      \item $\hat\beta_{\text{ols}} = (y_1 + 2y_2)/5$;
      \item $\Var(\hat\beta_{\text{ols}}\mid x) = 1/5$;
      \item Заметим, что по величине $2y_1 - y_2$ можно однозначно восстановить величины ошибок $u_1$ и $u_2$.
      Например, если $2y_1 -y_2 = 3$, то $u_1 = 1$, $u_2 = -1$.
      \[
        \hat\beta_{\text{best}} = \begin{cases}
          y_1 + 1, \text{ если } 2y_1 - y_2 < 0,\\
          y_1 - 1, \text{ если } 2y_1 - y_2 >0.
        \end{cases}
      \]
      \item Шок контент, $\Var(\hat\beta_{\text{best}}\mid x) = 0$.
      \item Построенная оценка $\hat\beta_{\text{best}}$ является нелинейной по $y$, 
      а теорема Гаусса — Маркова гарантирует только, что метод наименьших квадратов 
      порождает несмещённую оценку с наименьшей дисперсией среди линейных по $y$ оценок. 
    \end{enumerate}
        
  \end{sol}
\end{problem}

\begin{problem}(Hansen 4.14)

Задана модель $y = X \beta + u$, для которой выполняются предпосылки теоремы Гаусса~--- Маркова. Вас интересует величина $\theta = \beta^2$. Пусть получены МНК-оценки коэффициентов: $\hb$, $V_{\hb} = \Var [\hb | X]$. Тогда кажется неплохой идеей оценить $\theta$ как $\hat{\theta} = \hb^2$.

\begin{enumerate}
    \item Найдите $\mathbb{E} [\hat{\theta} | X]$. Является ли $\hat{\theta}$ смещённой?
    
    \item Предложите способ коррекции смещения для получения несмещённой оценки  $\hat{\theta}^*$, используя результаты предыдущего пункта.
\end{enumerate}

\end{problem}

\begin{problem}
Рассмотрим модель регрессии $y_i = \beta_1 + \beta_2x_{i2} + ... +\beta_kx_{ik} + \u_i$. Пусть все предпосылки теоремы Гаусса~--- Маркова выполнены. Дополнительно предположим, что $\u_i \sim N(0,\sigma^2), i=1,...,n.$. Дополнительно известно, что на самом деле $\beta_2 = ... = \beta_k = 0$.
\begin{enumerate}
\item Найдите $\E(R^2)$.
\item Найдите $\E(R^2_{\adj})$.
\end{enumerate}
\end{problem}

\subsection{Задачи для колаб:}

Генерация R2 для вывода распределения

Генерация смещения

Генерация лишних регерессоров

Реальный пример с лишним регрессорами (тип знаки зодиака и ретроградный)

Какая-то длинная задача, которую из темы в тему и в ней находить потом нарушения предпосылок?



https://colab.research.google.com/drive/1wFrLyGcVVETx96jS93I4z8asgAQwqIdw?usp=sharing

\subsection{Чёрный трэк:}

\textbf{Умножение блочных матриц.} Если размеры блоков допускают операцию умножения, то:

\[
\left[
\begin{array}{c|c}
A & B \\
\hline
C & D
\end{array}
\right]
\cdot
\left[
\begin{array}{c|c}
E & F \\
\hline
G & H
\end{array}
\right]
=
\left[
\begin{array}{c|c}
AE + BG &  AF+BH\\
\hline
CE+DG & CF+DH
\end{array}
\right].
\]

\bigskip

\textbf{Формула Фробениуса (блочное обращение).}
\[
\left[
\begin{array}{c|c}
A & B \\
\hline
C & D
\end{array}
\right]^{-1}=
\left[
\begin{array}{c|c}
A^{-1}+A^{-1}BH^{-1}CA^{-1} & -A^{-1}BH^{-1} \\
\hline
-H^{-1}CA^{-1} & H^{-1}
\end{array}
\right],
\]

где $A$ --- невырожденная квадратная матрица размерности $n \times n$, $D$ --- квадратная матрица размерности $k \times k$, $H = D - CA^{-1}B$.

\begin{problem}
Пусть истинной является модель $y = X_1\b_1+ X_2\b_2+u$, где $X_1$, $X_2$  --- матрицы признаков размерностей $n \times k_1$ и $n \times k_2$ соответственно. Вместо истинной модели вы оцениваете модель вида $y = X_1\b_1+v$, где $v$ --- вектор случайной ошибки, удовлетворяющий предпосылкам теоремы Гаусса~--- Маркова.
\begin{enumerate}
    \item Будет ли МНК-оценка вектора параметров $\b_1$ несмещённой?
    \item Будет ли несмещённой МНК-оценка дисперсии случайной ошибки?
    \item Рассчитайте $\Var(\hb_1)$. Не противоречит ли полученной результат теореме Гаусса~--- Маркова?
\end{enumerate}
\end{problem}

\begin{problem}
Пусть истинной является модель $y = X_1\b_1+u$, где $X_1$ --- матрица признаков размерности $n \times k_1$. Вместо истинной модели вы оцениваете модель вида $y = X_1\b_1+ X_2\b_2+v$, где $X_2$  --- матрица признаков размерности  $n \times k_2$, $v$ --- вектор случайной ошибки, удовлетворяющий предпосылкам теоремы Гаусса~--- Маркова.
\begin{enumerate}
    \item Будет ли МНК-оценка вектора параметров $\b_1$ несмещённой?
    \item Будет ли несмещённой МНК-оценка дисперсии случайной ошибки?
    \item Рассчитайте $\Var(\hb_1)$. Не противоречит ли полученной результат теореме Гаусса~--- Маркова?
\end{enumerate}
\end{problem}

\section{Доверительные интервалы для коэффициентов}
Построение доверительных интервалов для МНК оценок. Проверка гипотез. Асимптотика без нормальности ошибок. Нормальность ошибок.

\section{Бутстрэп}
Бутстрэп. Классический бутстрэп до регрессии и бутстрэп в регрессии. Метод наименьших модулей.

Чёрный трэк: возможно, разные варианты бутстрэпа в регрессии? BCA-бутстрэп до регрессии?

\section{Выбор функциональной формы}
Дамми-переменные и их интерпретация. Функциональные формы: полиномы, логарифмы, интерпретация коэффициентов. Информационные критерии.

Чёрный трэк: Структурные сдвиги. Тест Чоу. Локально-линейная регрессия (LOESS).

\section{Гетероскедастичность}
Гетероскедастичность. Тестирование гетероскедастичности. Робастные оценки. Доступный обобщённый МНК.

Задачи для доски:

Хансен: во сколько раз может быть недооценена дисперсия из-за гетероскедастичности


Коммент: акцент на робастных ошибках, тестирование и обобщённый МНК — кратко.

\section{Мультиколлинеарность и метод главных компонент}
Мультиколлинеарность и метод главных компонент.

Чёрный трэк: несколько взглядов на метод главных компонент? LASSO?


\section{Эндогенность}
Эндогенность. Инструментальные переменные. Ошибка измерения регрессора. Двухшаговый МНК.


\section{Эффекты воздействия}
Оценка эффектов воздействия. ATE. LATE. Четкий (sharp) и нечеткий (fuzzy) разрывный регрессионный дизайн (RDD).

Чёрный трэк: Метод разность разностей (DiD). Динамический метода разность разностей (Event Study).



\section{Логистическая регрессия: точечные оценки}
Логистическая регрессия: Бинарный и упорядоченный логит. Точечные оценки, прогнозы.  Интерпретация предельных эффектов.

Чёрный трэк: Множественные логиты. Неупорядоченные, условные, смешанные логиты.

\section{Логистическая регрессия: доверительные интервал}
Логистическая регрессия: доверительные интервалы ипроверка гипотез.




\end{document}

